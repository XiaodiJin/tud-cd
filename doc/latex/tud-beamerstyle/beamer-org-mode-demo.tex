% Created 2016-06-22 Mi 21:02
\documentclass[presentation,t]{beamer}
\usepackage[utf8]{inputenc}
\usepackage[T1]{fontenc}
\usepackage{fixltx2e}
\usepackage{graphicx}
\usepackage{longtable}
\usepackage{float}
\usepackage{wrapfig}
\usepackage{rotating}
\usepackage[normalem]{ulem}
\usepackage{amsmath}
\usepackage{textcomp}
\usepackage{marvosym}
\usepackage{wasysym}
\usepackage{amssymb}
\usepackage{hyperref}
\tolerance=1000
\pdfmapfile{+tud}%
\pdfmapfile{+tudscr}%
\usepackage{amsmath}
\usepackage{uniinput}
\usepackage{amsfonts}
\usepackage{tikz}
\usepackage[ngerman, germanb]{babel}
\usetikzlibrary{decorations.pathmorphing}
\usetheme[section,navigation,pagenum,ddc]{tud}
\useinnertheme[shadow=true]{rounded}
%\usetheme{Boadilla}
%\usecolortheme{tud}
\institut{Institut f\"ur Algebra}%
\DeclareMathOperator\Orb{Orb}%
\title[TUD-CD mit \LaTeX\ gesetzt]{Beamer-Präsentationen mit dem CD der TU Dresden}
\AtBeginDocument{\date{22.\,6.~2016}}
\subtitle{TUD-CD mit \LaTeX\ gesetzt}
\setbeamerfont{description item}{series=\bfseries}
\setbeamertemplate{date/place in footline}[default][T. Schlemmer]
\setbeamertemplate{page number in footline}[frame][total]
\AtBeginSection[]{\begin{frame}<beamer>[allowframebreaks]\frametitle{Abschnitt}\footnotesize\tableofcontents[currentsection]\end{frame}}
\setlength{\tudbeamerfooterplacewidth}{0.3\linewidth}%
\setlength{\tudbeamerfooterpagenumwidth}{5em}%
\makeatletter
\setlength{\tudbeamerfootertitlewidth}{\paperwidth-\beamer@leftmargin-\beamer@rightmargin
-\tudbeamerfooterplacewidth-\tudbeamerfooterpagenumwidth}%
\makeatother
\usetheme{default}
\author{Tobias Schlemmer}
\date{@latex:7.$\backslash$,5.\textasciitilde{}2012@}
\title{Beamer-Präsentationen mit dem CD der TU Dresden}
\hypersetup{
  pdfkeywords={},
  pdfsubject={},
  pdfcreator={Emacs 24.4.1 (Org mode 8.2.10)}}
\begin{document}

\maketitle
\begin{frame}[allowframebreaks]{Inhalt}
\tableofcontents
\end{frame}

\section{Einführung}
\label{sec-1}
\subsection{Allgemeines}
\label{sec-1-1}
\begin{frame}[label=sec-1-1-1]{Geschichte}
\begin{itemize}
\item Corporate Design der TU Dresden
\item tudbeamer Klasse (mit vielen Nachteilen)
\item tud-cd-Paket
\item \alert{Konfigurierbarer Beamer-Stil}
\end{itemize}
\end{frame}
\begin{frame}[label=sec-1-1-2]{CD-Vorgaben (Handbuch)}
\begin{itemize}
\item \textcolor{HKS65K100}{Ränder relativ zu Papiergrüße}
\item \textcolor{red}{Schriften in MS Punkt (inkompatible Maßeinheit)}
\item \textcolor{red}{Logoposition fehlt}
\end{itemize}
\end{frame}
\begin{frame}[label=sec-1-1-3]{Designkriterien}
\begin{itemize}
\item Interpretation der Vorgaben
\item Abweichend: HKS Farbschema und Univers
\item Kombinierbar mit \textbackslash useinnertheme\{rounded\}
\item Farbschema mit anderem Layout kombinierbar (partiell implementiert)
\item Alle Stile überschreibbar
\end{itemize}
\end{frame}
\subsection{Benutzung}
\label{sec-1-2}
\begin{frame}[label=sec-1-2-1]{Beispieldokument}
\textbackslash usepackage[ngerman]\{babel\}\\
\textbackslash usetheme[section,navigation]\{tud\}\\
\textbackslash institut\{Institut f\textbackslash "ur Algebra\}\%\\
\textbackslash title[Beispiel]\{Langer Titel mit Beispiel\}\\
\textbackslash begin\{document\}\\
\textbackslash maketitle\\
\textbackslash begin\{frame\}\{Eine Folie\}\\
Inhalt\\
\textbackslash end\{frame\}\\
\textbackslash end\{document\}\\
\end{frame}

\section{Einstellungen}
\label{sec-2}
\subsection{\textbackslash usetheme\{tud\}}
\label{sec-2-1}
\begin{frame}[label=sec-2-1-1]{Paketoptionen (Schriften)}
\begin{description}
\item[{noeulermath}] Benutze keine Euler-Mathematikschriften
\item[{noDIN}] Titel ohne DIN bold
\item[{serifmath}] Nutze Computer Modern für Mathematik
\item[{heavyfont}] Nutze Univers 55 statt Univers 45
\item[{beamerfont}] Keine Fontänderungen
\end{description}
\end{frame}
\begin{frame}[label=sec-2-1-2]{Paketoptionen (Seitenstruktur)}
\begin{description}
\item[{noheader}] Benutze Folien ohne Kopfzeile
\item[{smallrightmargin}] Schmaler rechter Rand (verletzt CD-Richtlinie
\item[{pagenum}] Seitenzahlen in der Fußzeile anzeigen
\item[{section}] mit Abschnittstitel auf jeder Folie
\item[{navbar}] mit Navigationshilfen
\item[{noddc}] ohne Dresden-Concept-Logo
\item[{ddc}] mit Dresden-Concept-Logo
\item[{ddcfooter}] mit Dresden-Concept Logo im Fußbereich (Standardeinstellung)
\end{description}
\end{frame}
\subsection{\textbackslash useoutertheme\{tud\}}
\label{sec-2-2}
\begin{frame}[label=sec-2-2-1]{Paketoptionen (\textbackslash useoutertheme\{tud\})}
\begin{description}
\item[{noheader}] Benutze Folien ohne Kopfzeile
\item[{smallrightmargin}] Schmaler rechter Rand (verletzt CD-Richtlinie
\item[{pagenum}] Seitenzahlen in der Fußzeile anzeigen
\item[{section}] mit Abschnittstitel auf jeder Folie
\item[{navbar}] mit Navigationshilfen
\item[{noddc}] ohne Dresden-Concept-Logo
\item[{ddc}] mit Dresden-Concept-Logo
\item[{ddcfooter}] mit Dresden-Concept Logo im Fußbereich (Standardeinstellung)
\end{description}
\end{frame}
\begin{frame}[label=sec-2-2-2]{Makros (\textbackslash useoutertheme\{tud\})}
\begin{description}
\item[{\textbackslash einrichtung\{\emph{einrichtung}\}}] Legt die
Einrichtung/Fakultät fest
\item[{\textbackslash fachrichtung\{\emph{Fachrichtung}\}}] Legt die
Fachrichtung fest
\item[{\textbackslash institut\{\emph{Institut}\}}] Legt das
Institut fest
\item[{\textbackslash professur\{\emph{Professur}\}}] Legt die
Professur fest
\item[{\textbackslash datecity\{\emph{Datumsort}\}}] Legt den Ort für den Vortrag fest
\end{description}
\end{frame}
\begin{frame}[label=sec-2-2-3]{Register (\textbackslash useoutertheme\{tud\})}
\begin{description}
\item[{\textbackslash topmarginnoheader}] Länge, die den Oberen Rand Seitenrand bei
Kopflosen Folien angibt.
\item[{\textbackslash tudbeamerfooterplacewidth}] Länge, die Breite des
Datums/Ort/(Autor-)Blockes in der Fußzeile bestimmt
\item[{\textbackslash tudbeamerfooterpagenumwidth}] Länge, die Platz für
die Seitenzahl in der Fußzeilereserviert
\item[{\textbackslash topmarginnoheader}] Länge, die Platz für den
Kurztitel in der Fußzeile reserviert
\end{description}
\end{frame}
\begin{frame}[allowframebreaks,label=sec-2-2-4]{Vorlagen (\textbackslash useoutertheme\{tud\})}
Vorlangen werden mit 
\textbf{\textbackslash setbeamertemplate\{\emph{Kategorie}\}[\emph{Vorlage}]}
gesetzt. Es können neue
definiert und ausgewählt werden. Weitere Informationen dazu gibt es im
„Beamer User Guide“ (\texttt{texdoc beamer}).
\begin{block}{headline}
Vorlage für Seitenkopf
\begin{description}
\item[{tud titlepage}] Titelseitenkopf
\item[{tud header}] normaler Folienkopf
\item[{tud noheader}] Folienkopf für kopflose Folien
\end{description}
\end{block}

\begin{block}{zweitlogo/titlepage}
Vorlage für ein zweites Logo im Kopf der Titelseite
\begin{description}
\item[{default}] kein Zweitlogo
\item[{ddc}] Dresden-Concept-Logo
\item[{logofile}] Eine Logo-Datei wird als Parameter übergeben und mit
\textbackslash includegraphics eingebunden und auf die
richtige Höhe skaliert.Beispiel: 
\textbf{\textbackslash setbeamertemplate\{zweitlogo/headline\}[logofile]\{fremdfirma.png\}}
\item[{freeform}] Es wird der \LaTeX -Code direkt als Parameter übergeben. Beispiel:
\textbf{\textbackslash setbeamertemplate\{zweitlogo/headline\}[freefrom]\{\LaTeX\}}
In diesem Fall ist die Höhe des Logos im Makro
\texttt{\textbackslash logoheight} hinterlegt. Für die
Weiterverarbeitung ist wichtig, dass es in ein
skaliertes Register expandiert wird (siehe
\texttt{beamerouterthemetod.sty}).
\end{description}
\end{block}

\begin{block}{zweitlogo/headline}
Vorlage für ein zweites Logo im Kopf einer normalen Seite (keine Titelseite)
\begin{description}
\item[{default}] kein Zweitlogo
\item[{ddc}] Dresden-Concept-Logo
\item[{logofile}] Eine Logo-Datei wird als Parameter übergeben und mit
\textbackslash includegraphics eingebunden und auf die
richtige Höhe skaliert. Beispiel: 
\textbf{\textbackslash setbeamertemplate\{zweitlogo/headline\}[logofile]\{fremdfirma.png\}}
\item[{freeform}] Es wird der \LaTeX -Code direkt als Parameter übergeben. Beispiel:
\textbf{\textbackslash setbeamertemplate\{zweitlogo/headline\}[freefrom]\{\LaTeX\}}
In diesem Fall ist die Höhe des Logos im Makro
\texttt{\textbackslash logoheight} hinterlegt. Für die
Weiterverarbeitung ist wichtig, dass es in ein
skaliertes Register expandiert wird (siehe
\texttt{beamerouterthemetod.sty}).
\end{description}
\end{block}

\begin{block}{date/place in footline}

\begin{description}
\item[{default}] Vorlage für Datum/Ort in Fußzeile mit optionalem Argument
für Ort. Standardwert „TU Dresden“.
\end{description}
\end{block}

\begin{block}{short title in footline}

\begin{description}
\item[{default}] Fügt den Kurztitel in die Fußzeile ein
\end{description}
\end{block}

\begin{block}{page number in footline}
\small
\begin{description}
\item[{page}] Setzt die Seitenzahl. Die Form kann als optionaler Parameter
übergeben werden. Standardwert „text and total". Für weitere Werte
siehe unten, Vorlagen, die mit „page number: “ anfangen (dieser Teil
wird automatisch hinzugefügt).
\item[{frame}] Setzt die Foliennummer. Die Form kann als optionaler Parameter
übergeben werden. Standardwert „text and total". Für weitere Werte
siehe unten, Vorlagen, die mit „frame number: “ anfangen (dieser Teil
wird automatisch hinzugefügt).
\end{description}
\end{block}

\begin{block}{page number: text and total}
Setzt Seitenzahl – Sprachauswahl für Nummer und Gesamtzahl.(wird automatisch gesetzt)
\begin{description}
\item[{english}] Englisch
\item[{german}] Deutsch
\end{description}
\end{block}

\begin{block}{page number: text}
Setzt Seitenzahl – Sprachauswahl für Nummer.(wird automatisch gesetzt)
\begin{description}
\item[{english}] Englisch
\item[{german}] Deutsch
\end{description}
\end{block}

\begin{block}{page number: total}
Setzt Seitenzahl und Gesamtzahl ohne Worte.(wird automatisch gesetzt)
\begin{description}
\item[{default}] Standardeinstellung
\end{description}
\end{block}

\begin{block}{page number: only}
Setzt Seitenzahl ohne Gesamtzahl ohne Worte.(wird automatisch gesetzt)
\begin{description}
\item[{default}] Standardeinstellung
\end{description}
\end{block}

\begin{block}{frame number: text and total}
Setzt Foliennummer – Sprachauswahl für Nummer und Gesamtzahl.(wird automatisch gesetzt)
\begin{description}
\item[{english}] Englisch
\item[{german}] Deutsch
\end{description}
\end{block}

\begin{block}{frame number: text}
Setzt Foliennummer – Sprachauswahl für Nummer.(wird automatisch gesetzt)
\begin{description}
\item[{english}] Englisch
\item[{german}] Deutsch
\end{description}
\end{block}

\begin{block}{frame number: total}
Setzt Foliennummer und Gesamtzahl ohne Worte.(wird automatisch gesetzt)
\begin{description}
\item[{default}] Standardeinstellung
\end{description}
\end{block}

\begin{block}{frame number: only}
Setzt Foliennummer ohne  Gesamtzahl ohne Worte.(wird automatisch gesetzt)
\begin{description}
\item[{default}] Standardeinstellung
\end{description}
\end{block}


\begin{block}{footline}
Setzt den Seitenfuß

\begin{description}
\item[{tud titlepage}] Fußzeile auf der Titelseite
\item[{tud pagenum}] Fußzeile mit Seiten- oder Folienzahl entsprechend dem
Parameter zu „page number in footline“
\item[{tud nopagenum}] Fußzeile ohne Seiten- und Folienzahl
\end{description}
\end{block}

\begin{block}{frametitle}
Setzt den Folientitel
\begin{description}
\item[{tud titlesection}] Setzt vor dem eigentlichen Titel den
\end{description}
Abschnittstitel
\begin{description}
\item[{tud notitlesection}] Es wird nur der
\end{description}
\end{block}

\begin{block}{einrichtung/titlepage}
Setzt die Einrichtung im Seitenkopf auf der Titelseite

\begin{description}
\item[{default}] normal
\item[{empty}] keine Ausgabe
\end{description}
\end{block}
\begin{block}{fachrichtung/titlepage}
Setzt die Fachrichtung im Seitenkopf auf der Titelseite

\begin{description}
\item[{default}] normal
\item[{empty}] keine Ausgabe
\end{description}
\end{block}
\begin{block}{intstitut/titlepage}
Setzt den Institutsnamen im Seitenkopf auf der Titelseite

\begin{description}
\item[{default}] normal
\item[{empty}] keine Ausgabe
\end{description}
\end{block}

\begin{block}{professur/titlepage}
Setzt die Professur im Seitenkopf auf der Titelseite

\begin{description}
\item[{default}] normal
\item[{empty}] keine Ausgabe
\end{description}
\end{block}
\end{frame}

\begin{frame}[label=sec-2-2-5]{Beispiel (\textbackslash useoutertheme\{tud\})}
\textbackslash setbeamercolor\{normal text\}\{bg=white\}\\
\textbackslash setbeamertemplate\{headline\}[tud header]\\
\textbackslash setbeamertemplate\{footline\}[tud pagenum]\\
\textbackslash setbeamertemplate\{frametitle\}[tud notitlesection]\\
\end{frame}


\subsection{\textbackslash usefonttheme\{tud\}}
\label{sec-2-3}

\begin{frame}[label=sec-2-3-1]{Paketoptionen (\textbackslash usefonttheme\{tud\})}
\begin{description}
\item[{noeulermath}] Benutze keine Euler-Mathematikschriften
\item[{noDIN}] Titel ohne DIN bold
\item[{nodin}] Titel ohne DIN bold
\item[{serifmath}] Nutze Computer Modern für Mathematik
\item[{heavyfont}] Nutze Univers 55 statt Univers 45
\item[{beamerfont}] Keine Fontänderungen
\end{description}
\end{frame}

\begin{frame}[label=sec-2-3-2]{Makros (\textbackslash usefonttheme\{tud\})}
\begin{description}
\item[{\textbackslash tudtitlenormalsize}] Ersatz für \textbackslash
  normalsize auf der Titelseite
\item[{\textbackslash tudtitlesmall}] Ersatz für \textbackslash small auf der Titelseite
\item[{\textbackslash tudtitletiny}] Ersatz für \textbackslash tiny auf der Titelseite
\end{description}

Darüberhinaus lädt dieses Paket das Paket „tudfonts“ mit all seinen makros
\end{frame}

\begin{frame}[allowframebreaks,label=sec-2-3-3]{Schriftvorlagen (\textbackslash usefonttheme\{tud\})}
Vorlangen werden mit \textbackslash
setbeamerfont\{\emph{Name}\}\{Werte\} gesetzt. Sie können mit \textbackslash
usebeamerfont aktiviert werden. Weitere Informationen dazu gibt es im
„Beamer User Guide“ (\texttt{texdoc beamer}).

Es werden folgende Vorlagen definiert:
\begin{itemize}
\item \textbackslash setbeamerfont\{itemize/enumerate subbody\} \{size=\textbackslash scriptsize\}
\item \textbackslash setbeamerfont\{itemize/enumerate subsubbody\} \{size=\textbackslash scriptsize\}
\item \textbackslash setbeamerfont\{section in head/foot\}\{size=\textbackslash normalsize, family=\textbackslash sffamily\}
\item \textbackslash setbeamerfont\{frametitle\} \{size=\textbackslash normalsize, family=\textbackslash sffamily\}
\item \textbackslash setbeamerfont\{framesubtitle\} \{size=\textbackslash
   small, series=\textbackslash bfseries,family=\textbackslash sffamily\}
\item \textbackslash setbeamerfont\{footline\} \{size=\textbackslash tiny\}
\item \textbackslash setbeamerfont\{block title\} \{size=\{\}\}
\item \textbackslash if@noDIN
  \textbackslash setbeamerfont\{title\} \{size=\textbackslash
 @setfontsize\textbackslash LARGE\textbackslash @xviipt\{22\},
 series=\textbackslash bfseries, family=\textbackslash sffamily\}
\textbackslash else
  \textbackslash setbeamerfont\{title\} \{size=\textbackslash
 @setfontsize\textbackslash LARGE\textbackslash @xviipt\{22\},
 series=\textbackslash bfseries, family=\textbackslash dinfamily\}
\textbackslash fi
\item \textbackslash setbeamerfont\{subtitle\} \{series=\textbackslash
   bfseries, family=\textbackslash sffamily\}
\item \textbackslash setbeamerfont\{einrichtung/titlepage\}
\{size=\textbackslash tudtitletiny, series=\textbackslash bfseries\}
\item \textbackslash setbeamerfont\{fachrichtung/titlepage\} \{size=\textbackslash tudtitletiny\}
\item \textbackslash setbeamerfont\{institut/titlepage\} \{parent=fachrichtung/titlepage\}
\item \textbackslash setbeamerfont\{professur/titlepage\} \{parent=fachrichtung/titlepage\}
\item \textbackslash setbeamerfont\{date in head/foot/titlepage\} \{size=\textbackslash tudtitlenormalsize\}
\item \textbackslash setbeamerfont\{author\} \{size=\textbackslash tudtitlesmall\}
\end{itemize}
\end{frame}



\subsection{\textbackslash usecolortheme\{tud\}}
\label{sec-2-4}

\begin{frame}[label=sec-2-4-1]{Makros (\textbackslash usecolortheme\{tud\})}
\begin{description}
\item[{\textbackslash darktitlepale}] Stellt eine dunkle Titelseite ein, wie
\end{description}
vom CD gefordert
\begin{description}
\item[{\textbackslash whitetitlepage}] Titelseite wird weiß (nicht CD-Konform).
\end{description}

Darüberhinaus lädt dieses Paket das Paket „tudcolors“ mit all seinen
makros und den HKS-Farben (siehe Dokumentation von tudmathposter)
\end{frame}

\begin{frame}[allowframebreaks,label=sec-2-4-2]{Farbvorlagen (\textbackslash usecolortheme\{tud\})}
Farbvorlagen werden mit \textbackslash
setbeamercolor\{\emph{Name}\}\{Werte\} gesetzt. Sie können mit \textbackslash
usebeamercolor für die Aktivierung geladen werden. Weitere Informationen dazu gibt es im
„Beamer User Guide“ (\texttt{texdoc beamer}).

Es werden folgende Vorlagen definiert:
\begin{itemize}
\item \textbackslash setbeamercolor\{normal text\} \{fg=HKS41K100,bg=white\}
\item \textbackslash setbeamercolor\{structure\} \{fg=HKS41K100\}
\item \textbackslash setbeamercolor\{alerted text\} \{fg=HKS44K100\}
\item \textbackslash setbeamercolor\{alternate palette\} \{fg=HKS92K80\}
\item \textbackslash setbeamercolor\{date in head/foot\} \{parent=alternate palette\}
\item \textbackslash setbeamercolor\{title in head/foot\} \{parent=alternate palette\}
\item \textbackslash setbeamercolor\{page number in head/foot\} \{parent=alternate palette\}
\item \textbackslash setbeamercolor\{section in head/foot\} \{parent=alternate palette\}
\item \textbackslash setbeamercolor\{subsection in head/foot\} \{parent=section in head/foot\}
\item \textbackslash setbeamercolor\{upper separation line head\} \{parent=alternate palette\}
\item \textbackslash setbeamercolor\{lower separation line head\} \{parent=upper separation line head\}
\item \textbackslash setbeamercolor\{author in head/foot\} \{parent=section in head/foot\}
\item \textbackslash setbeamercolor\{title in head/foot\} \{parent=subsection in head/foot\}
\item \textbackslash setbeamercolor\{logo\} \{use=structure, fg=structure.fg\}
\item \textbackslash newcommand*\{\textbackslash darktitlepage\}\{\%
\begin{itemize}
\item \textbackslash setbeamercolor\{normal text/titlepage\} \{fg=white,bg=HKS41K100\}\%
\item \textbackslash setbeamercolor\{title\} \{use=normal text/titlepage, fg=normal text/titlepage.fg\}\%
\item \textbackslash setbeamercolor\{subtitle\} \{use=normal
text/titlepage, fg=normal text/titlepage.fg\}\%
\item \textbackslash setbeamercolor\{author/titlepage\} \{use=normal
text/titlepage, fg=normal text/titlepage.fg\}\%
\item \textbackslash setbeamercolor\{headline/titlepage\} \{use=normal
text/titlepage, fg=normal text/titlepage.fg\}\%
\item \textbackslash setbeamercolor\{logo/titlepage\} \{use=normal
text/titlepage, fg=normal text/titlepage.fg\}\%
\item \textbackslash setbeamercolor\{einrichtung/titlepage\}
\{use=normal text/titlepage, fg=normal text/titlepage.fg\}\%
\item \textbackslash setbeamercolor\{fachrichtung/titlepage\}
\{use=einrichtung/titlepage, fg=einrichtung/titlepage.fg\}\%
\item \textbackslash setbeamercolor\{institut/titlepage\}
\{use=einrichtung/titlepage, fg=einrichtung/titlepage.fg\}\%
\item \textbackslash setbeamercolor\{professur/titlepage\}
\{use=einrichtung/titlepage, fg=einrichtung/titlepage.fg\}\%
\item \textbackslash setbeamercolor\{upper separation line
head/titlepage\} \{use=normal text/titlepage, fg=normal text/titlepage.fg\}\%
\item \textbackslash setbeamercolor\{lower separation line head/titlepage\}\%
\{use=upper separation line head/titlepage, fg=upper separation line head/titlepage.fg\}\%
\item \textbackslash setbeamercolor\{date in head/foot/titlepage\}
\{use=normal text/titlepage, fg=normal text/titlepage.fg\}\%
\item \textbackslash let\textbackslash logo$\backslash$@ DDC\textbackslash logo$\backslash$@ DDC$\backslash$@ white
\item \textbackslash let\textbackslash logo$\backslash$@ DDCf\textbackslash logo$\backslash$@ DDC$\backslash$@ whitef
\}
\end{itemize}
\item \textbackslash newcommand*\{\textbackslash whitetitlepage\}\{\%
\begin{itemize}
\item \textbackslash setbeamercolor\{normal text/titlepage\}
\{use=normal text,fg=normal text.fg, bg=normal text.bg\}
\item \textbackslash setbeamercolor\{title\} \{use=normal
text/titlepage, fg=normal text/titlepage.fg\}\%
\item \textbackslash setbeamercolor\{subtitle\} \{use=normal
text/titlepagexo, fg=normal text/titlepage.fg\}\%
\item \textbackslash setbeamercolor\{author/titlepage\} \{use=normal
text/titlepage, fg=normal text/titlepage.fg\}\%
\item \textbackslash setbeamercolor\{headline/titlepage\} \{use=normal
text/titlepage, fg=normal text/titlepage.fg\}\%
\item \textbackslash setbeamercolor\{logo/titlepage\} \{use=normal
text/titlepage, fg=normal text/titlepage.fg\}\%
\item \textbackslash setbeamercolor\{einrichtung/titlepage\} \{fg=HKS92K100\}\%
\item \textbackslash setbeamercolor\{fachrichtung/titlepage\}
\{use=einrichtung/titlepage, fg=einrichtung/titlepage.fg\}\%
\item \textbackslash setbeamercolor\{institut/titlepage\}
\{use=einrichtung/titlepage, fg=einrichtung/titlepage.fg\}\%
\item \textbackslash setbeamercolor\{professur/titlepage\}
\{use=einrichtung/titlepage, fg=einrichtung/titlepage.fg\}\%
\item \textbackslash setbeamercolor\{upper separation line
head/titlepage\} \{fg=HKS92K100\}\%
\item \textbackslash setbeamercolor\{lower separation line head/titlepage\}\%
\{use=upper separation line head/titlepage,fg=upper separation line
  head/titlepage.fg\}\%
\item \textbackslash setbeamercolor\{date in head/foot/titlepage\} \{fg=HKS92K100\}\%
\item \textbackslash let\textbackslash logo$\backslash$@ DDC\textbackslash logo$\backslash$@ DDC$\backslash$@ bunt\%
\item \textbackslash let\textbackslash logo$\backslash$@ DDCf\textbackslash logo$\backslash$@ DDC$\backslash$@ colorf
\}
\end{itemize}
\item \textbackslash darktitlepage

\item \textbackslash setbeamercolor\{block body\} \{use=normal text,
fg=normal text.fg, bg=HKS41K10\}
\item \textbackslash setbeamercolor\{block title\} \{fg=HKS41K100,bg=HKS41K20\}
\item \textbackslash setbeamercolor\{block body example\} \{use=normal
text, fg=normal text.fg, bg=HKS41K10\}
\item \textbackslash setbeamercolor\{block title example\}
\{fg=HKS57K100, bg=HKS41K20\}
\item \textbackslash setbeamercolor\{block body alerted\} \{use=normal
text, fg=normal text.fg, bg=HKS65K10\}
\item \textbackslash setbeamercolor\{block title alerted\}
\{fg=HKS07K100, bg=HKS65K20\}
\end{itemize}
\end{frame}



\subsection{\textbackslash useinnertheme\{tud\}}
\label{sec-2-5}

\begin{frame}[allowframebreaks,label=sec-2-5-1]{Vorlagen (\textbackslash useinnertheme\{tud\})}
Für die inneren Vorlagen gibt es keine Vorgaben außer für die
Titelseite.
\begin{block}{Verwendete Vorlagen/Makros}
\begin{description}
\item[{\textbackslash beamertemplatedotitem}] siehe beamer-Dokumentation
\item[{\textbackslash usesubitemizeitemtemplate\{-$\backslash$/-\}}] siehe
beamer-Dokumentation
\item[{\textbackslash setbeamertemplate\{title page\}[tud]}] stellt das CD
der TUD für die Titelseite ein. Dieses Makro wird automatisch mit
\textbackslash begin \{document\} aufgerufen.
\end{description}
\end{block}
\end{frame}

\section{Tips und Tricks}
\label{sec-3}
\subsection{Tips}
\label{sec-3-1}
\begin{frame}[label=sec-3-1-1]{Fußzeile}
Die Fußzeile sollte die wichtigsten Informationen enthalten, an die
sich das Publikum erinnern soll. Wenn das TU-Logo im Kopf enthalten
ist die Angabe „TU Dresden“ redundant.

\textbackslash setbeamertemplate\{date/place in
footline\}[default][I. Nachname] 

Setzt statt der TU Dresden den Namen
des Sprechers.
\end{frame}
\begin{frame}[label=sec-3-1-2]{Blöcke mit runden Ecken und Schatten}
\begin{block}{Aufruf}
Blöcke wie dieser hier werden mittels 

\textbackslash useinnertheme[shadow=true]\{rounded\}

in der Dokumentpräambel voreingestellt
\end{block}
\end{frame}
\subsection{Ein Beispiel}
\label{sec-3-2}
\begin{frame}[label=sec-3-2-1]{Eine Beispielfolie}
\begin{columns}
\begin{column}{0.3\textwidth}
\tikzstyle{automorphismuspfeil}=[->,HKS44K100,decoration={bent,aspect=0.3,amplitude=3},decorate]
\tikzstyle{automorphismuspfeil2}=[automorphismuspfeil,decoration={bent,aspect=0.3,amplitude=2}]
\begin{tikzpicture}
\draw (0,0) node (n0) {0};
\draw (-1,1) node (n1) {1};
\draw (1,1) node (n2) {2};
\draw (-1.5,2) node (n3) {3};
\draw (-0.5,2) node (n4) {4};
\draw (0.5,2) node (n5) {5};
\draw (1.5,2) node (n6) {6};
\draw (0,3) node (n7) {7};
\draw (n0) -- (n1) -- (n3) -- (n7) -- (n6) -- (n2) -- (n0);
\draw (n1) -- (n4) -- (n7) -- (n5) -- (n2);
\only<1,2,5,6> {
  \draw[automorphismuspfeil] (n1) -- (n2);
  \draw[automorphismuspfeil] (n2) -- (n1); 
}
\only<3,4>{
  \draw[automorphismuspfeil] (n3) -- (n4);
  \draw[automorphismuspfeil] (n4) -- (n3);
}
\only<5>{
  \draw[automorphismuspfeil] (n3) -- (n5);
  \draw[automorphismuspfeil] (n5) -- (n3);
}
\only<6>{
  \draw[automorphismuspfeil] (n3) -- (n6);
  \draw[automorphismuspfeil] (n6) -- (n3);
}
\only<6>{
  \draw[automorphismuspfeil2] (n4) -- (n5);
  \draw[automorphismuspfeil2] (n5) -- (n4);
}
\only<5>{
  \draw[automorphismuspfeil] (n4) -- (n6);
  \draw[automorphismuspfeil] (n6) -- (n4);
}
\only<3,4>{
  \draw[automorphismuspfeil] (n5) -- (n6);
  \draw[automorphismuspfeil] (n6) -- (n5);
}
\only<1>{
  \draw[automorphismuspfeil] (n3) -- (n5);
  \draw[automorphismuspfeil2] (n5) -- (n4);
  \draw[automorphismuspfeil] (n4) -- (n6);
  \draw[automorphismuspfeil] (n6) -- (n3);
}
\only<2>{
  \draw[automorphismuspfeil] (n3) -- (n6);
  \draw[automorphismuspfeil] (n6) -- (n4);
  \draw[automorphismuspfeil2] (n4) -- (n5);
  \draw[automorphismuspfeil] (n5) -- (n3);
}
\end{tikzpicture}
\end{column}
\begin{column}{0.7\textwidth}
\begin{block}{Beschreibung}

Eine Tabelle

{}\uncover<2->{\tiny %}
\begin{center}
\begin{tabular}{lllllllll}
\hline
 & $1$ & $a$ & $b$ & $c$ & $d$ & $e$ & $f$ & $g$\\
\hline
$1=(1)$ & $1$ & $a$ & $b$ & $c$ & $d$ & $e$ & $f$ & $g$\\
$a=(34)$ & $a$ & $1$ & $c$ & $b$ & $g$ & $f$ & $e$ & $d$\\
$b=(56)$ & $b$ & $c$ & $1$ & $a$ & $f$ & $g$ & $d$ & $e$\\
$c=(34)(56)$ & $c$ & $b$ & $a$ & $1$ & $e$ & $d$ & $g$ & $f$\\
$d=(12)(35)(46)$ & $d$ & $f$ & $g$ & $e$ & $1$ & $c$ & $a$ & $b$\\
$e=(12)(36)(45)$ & $e$ & $g$ & $f$ & $d$ & $c$ & $1$ & $b$ & $a$\\
$f=(12)(3546)$ & $f$ & $d$ & $e$ & $g$ & $b$ & $a$ & $c$ & $1$\\
$g=(12)(3645)$ & $g$ & $e$ & $d$ & $f$ & $a$ & $b$ & $1$ & $c$\\
\hline
\end{tabular}
\end{center}
}
\end{block}
\end{column}
\end{columns}
\end{frame}


\section{Minimal agierende Automorphismen}
\label{sec-4}
\subsection{Quasiordnung der Automorphismen}
\label{sec-4-1}
\begin{frame}[label=sec-4-1-1]{Ein Satz mit Beweis}
\begin{theorem}[Quasiordnung]
Sei $G\leq \mathfrak{Aut}(M,\leq)$ eine Untergruppe der geordneten
Menge $(M,\leq)$. Dann ist die Relation ${\sqsubseteq}\subseteq
G\times G$ mit 
\[
g\sqsubseteq h :\Leftrightarrow ∀u∈\Orb(〈g〉)∃U'∈\Orb(〈h〉)\text{ und }U\subseteq U'
\]
eine Quasiordnung.
\end{theorem}

\begin{proof}[Beweis]
\begin{enumerate}
\item Reflexiv: $\Orb(〈g〉) = \Orb(〈g〉)$
\item Transitiv:
\end{enumerate}
\[
\begin{matrix}∀U∈\Orb(〈g〉)∃U'∈\Orb(〈h〉): U\subseteq U'\\
  ∀V∈\Orb(〈h〉)∃V'∈\Orb(〈i〉): V\subseteq V'
\end{matrix}\quad ⇒\quad ∃U''∈\Orb(〈f〉): U\subseteq U''.
\]\vspace{-0.25\baselineskip}
\end{proof}
\end{frame}
\begin{frame}[label=sec-4-1-2]{Beispielfolie}
\begin{example}[Beispieltitel]
mal sehen, was wird
\end{example}
\begin{alertblock}{Block hervorgehoben}
Inhalt
\end{alertblock}
\end{frame}
% Emacs 24.4.1 (Org mode 8.2.10)
\end{document}